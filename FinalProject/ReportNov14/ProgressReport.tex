%%%%%%%%%%%%%%%%%%%%%%%%%%%%%%%%%%%%%%%%%%%%%%%%%%%%%%%%%%%%%%%%%%
%%%%%%%% ICML 2014 EXAMPLE LATEX SUBMISSION FILE %%%%%%%%%%%%%%%%%
%%%%%%%%%%%%%%%%%%%%%%%%%%%%%%%%%%%%%%%%%%%%%%%%%%%%%%%%%%%%%%%%%%

% Use the following line _only_ if you're still using LaTeX 2.09.
%\documentstyle[icml2014,epsf,natbib]{article}
% If you rely on Latex2e packages, like most moden people use this:
\documentclass{article}

% use Times
\usepackage{times}
% For figures
\usepackage{graphicx} % more modern
%\usepackage{epsfig} % less modern
\usepackage{subfigure} 

% For citations
\usepackage{natbib}

% For algorithms
\usepackage{algorithm}
\usepackage{algorithmic}

% As of 2011, we use the hyperref package to produce hyperlinks in the
% resulting PDF.  If this breaks your system, please commend out the
% following usepackage line and replace \usepackage{icml2014} with
% \usepackage[nohyperref]{icml2014} above.
\usepackage{hyperref}

% Packages hyperref and algorithmic misbehave sometimes.  We can fix
% this with the following command.
\newcommand{\theHalgorithm}{\arabic{algorithm}}

% Employ the following version of the ``usepackage'' statement for
% submitting the draft version of the paper for review.  This will set
% the note in the first column to ``Under review.  Do not distribute.''
\usepackage[accepted]{icml2014} 


% The \icmltitle you define below is probably too long as a header.
% Therefore, a short form for the running title is supplied here:
\icmltitlerunning{Jing, Xu, Tian, Huang}

\begin{document} 

\twocolumn[
\icmltitle{Stock price prediction with LSTM network\\Project Progress Report for COMP 562}

% It is OKAY to include author information, even for blind
% submissions: the style file will automatically remove it for you
% unless you've provided the [accepted] option to the icml2014
% package.
\icmlauthor{Siyang Jing}{siyangj@live.unc.edu}
\icmlauthor{Jiyu Xu}{xujiyu@live.unc.edu}
\icmlauthor{Jiacheng Tian}{jctian@live.unc.edu}
\icmlauthor{Yuhui Huang}{yuhui97@live.unc.edu}

% You may provide any keywords that you 
% find helpful for describing your paper; these are used to populate 
% the "keywords" metadata in the PDF but will not be shown in the document
\icmlkeywords{boring formatting information, machine learning}

\vskip 0.3in
]

\begin{abstract} 
The time series of stock prices are non-stationary and nonlinear, making the prediction of future price trends much challenging. Inspired by Convolutional Neural Network (CNN), we make convolution
on the time dimension to capture the long-term fluctuation features of
stock series. To learn long-term dependencies of stock prices, we combine
the time convolution with Long Short-Term Memory (LSTM), and propose
a novel deep learning model named Time Convolution Long ShortTerm
Memory (TC-LSTM) networks. TC-LSTM can obtain the stock
longer data dependence and overall change pattern. The experiments on
two real market datasets demonstrate that the proposed model outperforms
other three baseline models in the mean square error.
\end{abstract} 


\section{Introduction}

In the financial industry, stock price prediction has constantly been a popular field of research, because stock price predictability is one of the most important concerns for investors. According to many widely accepted studies, the financial markets, and particularly stock markets, have been proved to be predictable in some scenarios. While different features are available for prediction, such as general economic climate and social media effects, most studies focused on analysis of past trading patterns. 

\subsection{Literature Review}

Our research assumes minute-level price fluctuation pattern is independent of corporate fundamentals and macro economy. Thus, unlike the studies of \cite{Chiang2016}, \cite{Chourmouziadis2016}, and \cite{Zhong2017} in which daily price data are used as input, we seek to develop a predictive model based on minute-level input price data. The prediction of future stock price had also been understood as both classification and regression problems in previous studies. \cite{Chen2016} and \cite{Zhong2017} provided prediction of market direction as either up or down. In more complicated cases, \cite{Chourmouziadis2016} specified cash and stock within the optimal portfolio composition. Our study intends to give a prediction of the stock return in the next minute compared to the current time point.

There have been linear and nonlinear models to predict stock price movement with varying degrees of success. \cite{Chong2017} noted a multilayer artificial neural network might be particularly suitable with such time-series data, due to its higher computational power and sophistication of algorithm. Such model selects features based on raw input price data automatically and does not require understanding or providing data from the side of fundamentals or macro economy, which fits our assumption about minute-level price fluctuation pattern. Our model will be composed of two parts, with the first part being unsupervised learning with traditional ML techniques like RBM, PCA, etc. The second part takes advantage of recurrent neural network (RNN) model, especially its variant LSTM. For performance measurement, previous studies have used trade simulation or various MSE methods \cite{Chiang2016}; \cite{Chourmouziadis2016}; \cite{Zhong2017}; \cite{Chong2017}. We plan to use MSE in the assessment. 

\subsection{Data}

Due to certain limitation and just for preliminary testing of our strategy, we are currently using 10 days of minute-level price data of 50 stocks, but our aim is to obtain data of 10 years for model training. The input will be 500 dimensional lagged stock returns, which are the returns of 50 stocks in the previous 10 minutes. We may adjust the number of lagged periods for better performance later. We use the first 9 days as training, and the last day as test.

\section{Method} 

\subsection{Figures}
 
You may want to include figures in the paper to help readers visualize
your approach and your results. Such artwork should be centered,
legible, and separated from the text. Lines should be dark and at
least 0.5~points thick for purposes of reproduction, and text should
not appear on a gray background.

Label all distinct components of each figure. If the figure takes the
form of a graph, then give a name for each axis and include a legend
that briefly describes each curve. Do not include a title inside the
figure; instead, be sure to include a caption describing your figure.

You may float figures to the top or
bottom of a column, and you may set wide figures across both columns
(use the environment {\tt figure*} in \LaTeX), but always place
two-column figures at the top or bottom of the page.

\subsection{Algorithms}

If you are using \LaTeX, please use the ``algorithm'' and ``algorithmic'' 
environments to format pseudocode. These require 
the corresponding stylefiles, algorithm.sty and 
algorithmic.sty, which are supplied with this package. 
Algorithm~\ref{alg:example} shows an example. 

\begin{algorithm}[tb]
   \caption{Bubble Sort}
   \label{alg:example}
\begin{algorithmic}
   \STATE {\bfseries Input:} data $x_i$, size $m$
   \REPEAT
   \STATE Initialize $noChange = true$.
   \FOR{$i=1$ {\bfseries to} $m-1$}
   \IF{$x_i > x_{i+1}$} 
   \STATE Swap $x_i$ and $x_{i+1}$
   \STATE $noChange = false$
   \ENDIF
   \ENDFOR
   \UNTIL{$noChange$ is $true$}
\end{algorithmic}
\end{algorithm}
 
\subsection{Tables} 
 
You may also want to include tables that summarize material. Like 
figures, these should be centered, legible, and numbered consecutively. 
However, place the title {\it above\/} the table, as in 
Table~\ref{sample-table}.
% Note use of \abovespace and \belowspace to get reasonable spacing 
% above and below tabular lines. 

\begin{table}[t]
\caption{Classification accuracies for naive Bayes and flexible 
Bayes on various data sets.}
\label{sample-table}
\vskip 0.15in
\begin{center}
\begin{small}
\begin{sc}
\begin{tabular}{lcccr}
\hline
\abovespace\belowspace
Data set & Naive & Flexible & Better? \\
\hline
\abovespace
Breast    & 95.9$\pm$ 0.2& 96.7$\pm$ 0.2& $\surd$ \\
Cleveland & 83.3$\pm$ 0.6& 80.0$\pm$ 0.6& $\times$\\
Glass2    & 61.9$\pm$ 1.4& 83.8$\pm$ 0.7& $\surd$ \\
Credit    & 74.8$\pm$ 0.5& 78.3$\pm$ 0.6&         \\
Horse     & 73.3$\pm$ 0.9& 69.7$\pm$ 1.0& $\times$\\
Meta      & 67.1$\pm$ 0.6& 76.5$\pm$ 0.5& $\surd$ \\
Pima      & 75.1$\pm$ 0.6& 73.9$\pm$ 0.5&         \\
\belowspace
Vehicle   & 44.9$\pm$ 0.6& 61.5$\pm$ 0.4& $\surd$ \\
\hline
\end{tabular}
\end{sc}
\end{small}
\end{center}
\vskip -0.1in
\end{table}

Tables contain textual material that can be typeset, as contrasted 
with figures, which contain graphical material that must be drawn. 
Specify the contents of each row and column in the table's topmost
row. Again, you may float tables to a column's top or bottom, and set
wide tables across both columns, but place two-column tables at the
top or bottom of the page.
 
\subsection{Citations and References} 

Please use APA reference format regardless of your formatter
or word processor. If you rely on the \LaTeX\/ bibliographic 
facility, use {\tt natbib.sty} and {\tt icml2014.bst} 
included in the style-file package to obtain this format.

Citations within the text should include the authors' last names and
year. If the authors' names are included in the sentence, place only
the year in parentheses, for example when referencing Arthur Samuel's
pioneering work \yrcite{Samuel59}. Otherwise place the entire
reference in parentheses with the authors and year separated by a
comma \cite{Samuel59}. List multiple references separated by
semicolons \cite{kearns89,Samuel59,mitchell80}. Use the `et~al.'
construct only for citations with three or more authors or after
listing all authors to a publication in an earlier reference \cite{MachineLearningI}.

The references at the end of this document give examples for journal
articles \cite{Samuel59}, conference publications \cite{langley00}, book chapters \cite{Newell81}, books \cite{DudaHart2nd}, edited volumes \cite{MachineLearningI}, 
technical reports \cite{mitchell80}, and dissertations \cite{kearns89}. 

Alphabetize references by the surnames of the first authors, with
single author entries preceding multiple author entries. Order
references for the same authors by year of publication, with the
earliest first. Make sure that each reference includes all relevant
information (e.g., page numbers).

\section*{Acknowledgments} 
 
If you did this work in collaboration with someone else, or if someone else (such as another professor) had advised you on this work, your report must fully acknowledge their contributions. If you received external help or assistance on this project, you must cite these sources here in the acknowledgements section.  If you do not have anything to list in this section, write simply ``None.''

\bibliography{../Papers/Reference}
\bibliographystyle{icml2014}

\end{document} 